\section{Milner-Barry Gambit}
\newchessgame
\mainline{1. e4 e6 2. d4 d5 3. e5 c5 4. c3 Nc6 5. Nf3 Qb6 6. Bd3}

\chessboard[showmover=false,labelleft=false,labelright=false,labeltop=false,labelbottom=false]

The starting position of the Milner Barry Gambit. A popular option against the French,
white gets good development and central control in exchange for a pawn.

\subsection{8. cxd4 Nxd4 mainline}
\mainline{6...cxd4 7. O-O Bd7 8. cxd4 Nxd4}

\subsubsection{9. Ng5?!}
\mainline{9. Ng5}

\chessboard[showmover=false,labelleft=false,labelright=false,labeltop=false,labelbottom=false]

White makes the dual threats of Nxf7 / Nxh7. But there is nothing to worry about.
The best move is \mainline{9...Nc6}, attacking the e5 pawn

\paragraph{10. Nxf7} 
 
\mainline{10. Nxf7?? Kxf7 11. Qh5+ g6 12. Bxg6+ hxg6 13. Qxh8 Bg7 14. Qh3 Nxe5}

\chessboard[showmover=false,labelleft=false,labelright=false,labeltop=false,labelbottom=false]

and black is winning (-6)

\paragraph{10. Nxh7}

\newchessgame[setfen={r3kbnr/pp1b1ppN/1qn1p3/3pP3/8/3B4/PP3PPP/RNBQ1RK1 b kq - 0 10}, mover=b, moveid=10b]
\chessboard[showmover=false,labelleft=false,labelright=false,labeltop=false,labelbottom=false]

White grabs a pawn, but black can win the e5 pawn at his leisure and has a 
wicked attack along the h-file.

For example:
\mainline{10...Bc5}
Black prepares to castle queenside and not allow Nxf8.
White has mostly tried Qh5 and Nc3 on lichess. \variation{11. Qh5} 
loses immediatetly to \variation{11...O-O-O 12. Qxf7? Nxe5 13. Qxg7 Nxd3 14. Qxh8 Bxf2+ 15. Kh1 Bd4}

\paragraph{10. Re1}
\newchessgame[setfen={r3kbnr/pp1b1ppp/1qn1p3/3pP1N1/8/3B4/PP3PPP/RNBQR1K1 b kq - 3 10}, mover=b, moveid=10b]
\chessboard[showmover=false,labelleft=false,labelright=false,labeltop=false,labelbottom=false]

White defends e5

\mainline{10... Bc5} Hits f2 \mainline{11. Qf3} Hits f7 
\mainline{11...O-O-O} f7 is taboo as Qxf7 and Nxf7 can be met with Rf8 when black is crushing
\mainline{12. Nc3 Nge7} here white has only played one move on lichess: \mainline{13. Na4}

\chessboard[showmover=false,labelleft=false,labelright=false,labeltop=false,labelbottom=false]

It looks dangerous, but can be squashed by \mainline{13...Qa5} hitting e1 and a4
\mainline{14. Qd1 Bb4 15. Re2 Nxe5 16. Rxe5 Bxa4}

\chessboard[showmover=false,labelleft=false,labelright=false,labeltop=false,labelbottom=false]

Black has a very comfortable position (-2)

\subsubsection{9. Nxd4 mainline}

\newchessgame[setfen={r3kbnr/pp1b1ppp/1q2p3/3pP3/3N4/3B4/PP3PPP/RNBQ1RK1 b kq - 0 9}, mover=b, moveid=9b]
\chessboard[showmover=false,labelleft=false,labelright=false,labeltop=false,labelbottom=false]

\mainline{9...Qxd4 10. Nc3 a6} stops all Nb5 business

\paragraph{11. Qe2}

\mainline{11. Qe2 Ne7}

\subparagraph{12. Kh1}

\newchessgame[setfen={r3kb1r/1p1bnppp/p3p3/3pP3/3q4/2NB4/PP2QPPP/R1B2R1K b kq - 3 12}, mover=b, moveid=12b]
\chessboard[showmover=false,labelleft=false,labelright=false,labeltop=false,labelbottom=false]

\mainline{12...Bc6}
Almost a novelty (1 game played in the lichess master database)
\mainline{13. f4 Qb6 14. Be3 d4 15. Bf2 Nf5 16. Bxf5 exf5 17. e6 f6}

\subparagraph{12. Rd1}

\newchessgame[setfen={r3kb1r/1p1bnppp/p3p3/3pP3/3q4/2NB4/PP2QPPP/R1BR2K1 b kq - 3 12}, mover=b, moveid=12b]
\chessboard[showmover=false,labelleft=false,labelright=false,labeltop=false,labelbottom=false]

\mainline{12...Qh4} May look strange, but white cannot exploit the queen's positioning
\mainline{13. g3 Qh3 14. Qf3 h5} Be3 can be met with Nf5.
Nc6 is also good now that white doesn't have f4.

\paragraph{11. Re1}






\section{Classical French}

\subsection{6. Bxf6}

This move can be dangerous. White plans to play Bd3, e5, h4, and a Greek Gift with Bxh7+. It does not work, but black has to be careful.

\subsection{6. Bxe7}

White trades black's best piece. However, black's queen and d7 knight are developed. Black can quickly strike with c5 and f6. 

\subsection{7. f4}

I used to play 7. a6, but I believe 7. 0-0 is stronger. a6 is only useful if white castles queenside (as black can go b5).
If white castles kingside, then a6 is not very useful. In either case, black will need to castle kingside.
So it makes more sense to play castling first, and then only play a6 if white castles queenside.

\subsubsection{9. Qd2}

White probably will castle queenside. Now black can revert to the a6, b5, b4, a5, Ba6 plan.

\subsubsection{9. dxc5}

\subsubsection{9. Bd3}

\subsubsection{9. Nb5}

\subsection{7. Nb5}

There is an interesting sideline where white plays for c3 and Nc2, with perfectly harmonious pieces.
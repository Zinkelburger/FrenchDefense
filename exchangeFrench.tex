\section{Exchange Variation}

The Exchange French has a few different move orders, but the main one is

\newchessgame
\noindent \mainline{1. e4 e6 2. d4 d5 3. exd5 exd5}

\chessboard[showmover=false,labelleft=false,labelright=false,labeltop=false,labelbottom=false]

\noindent White has a few different options here, but there are only two kinds of approaches: white plays for c4, or white plays without c4.

\mainline{4. Nf3} The most common move in the position. The best scoring move and the one I recommend is \mainline{4...Nc6}.

\chessboard[showmover=false,labelleft=false,labelright=false,labeltop=false,labelbottom=false]

Black wants to play Bd6, Nge7, Bf5, and Nxf5. However, if black starts with Bd6 then white 
can play \variation{5. c4}, when black is no longer able to play \variation{5...Bb4+} in one move. 
So \variation{5...Nc6} is like a waiting move.

White can play \variation{5. Bb5}, \variation{5. Bd3}, \variation{5. Bf4}, or \variation{5. Nc3}

\subsection{5. Bb5}
\variation{5. Bb5}:
\newchessgame[setfen={r1bqkbnr/ppp2ppp/2n5/1B1p4/3P4/5N2/PPP2PPP/RNBQK2R b KQkq - 3 5}, mover=b, moveid=5b]

\noindent White prepares to castle and pins the c6 knight. Black should do likewise and prioritize castling.

\mainline{5...Bd6}

\chessboard[showmover=false,labelleft=false,labelright=false,labeltop=false,labelbottom=false]

\noindent This is the best move, preparing Nge7.
White's only try is

\mainline{6. c4!? dxc4 7. d5}

\chessboard[showmover=false,labelleft=false,labelright=false,labeltop=false,labelbottom=false]

\noindent Of course black should play \mainline{7...a6} 

if \variation{8. Bxc4} then \variation{8...Qe7+} when black either 
trades queens or plays Ne5.

\noindent Back to the mainline:

\mainline{8. Ba4 b5}
\mainline{9. dxc6 bxa4}

\chessboard[showmover=false,labelleft=false,labelright=false,labeltop=false,labelbottom=false]

Here if \variation{10. Qxa4?!} black is better with \variation{10...Qe7+ 11. Be3 Bc5 12. O-O Bxe3}

\noindent Back to the mainline:

\mainline{10. O-O Nge7}
\mainline{11. Nbd2}
hits c4
\mainline{11...Be6}
Defends c4

\chessboard[showmover=false,labelleft=false,labelright=false,labeltop=false,labelbottom=false]

\mainline{12. Qxa4 c3!}

\variation{12...Bd5} planning \variation{13...Bxc6} is the more popular line, but white is slightly better there. 
It is better to throw in c3 and only then play Bd5.

\mainline{13. bxc3 Bd5 14. Nd4 O-O}

Here \variation{15. c4?!} is the most popular move. 

But black is better with:

\chessboard[showmover=false,labelleft=false,labelright=false,labeltop=false,labelbottom=false]

\variation{15...Bxc6 16. Nxc6 Nxc6 17. Qxc6?? Qf6!}
And the dual threats of Bxh2+ and Qxa1 mean black is winning.

\variation{13. Ne4} is also met by \variation{13...Bd5}

\variation{11. Qxa4 Rb8}
I think \variation{11...O-O} will transpose

\newchessgame[setfen={1rbqk2r/2p1nppp/p1Pb4/8/Q1p5/5N2/PP3PPP/RNB2RK1 w k - 1 12}, mover=w, moveid=12w]
\chessboard[showmover=false,labelleft=false,labelright=false,labeltop=false,labelbottom=false]

\subsubsection{12. Qxc4?!}
The most popular move on lichess, but it gives black good development.

\mainline{12. Qxc4?! O-O 13. Nc3?! Rb4 14. Qe2 Nxc6 15. Bg5}

\chessboard[showmover=false,labelleft=false,labelright=false,labeltop=false,labelbottom=false]

Here black has only played \variation{15...f6?} on lichess.
But black can get a sizeable advantage (-1.1) with 
\mainline{15...Nd4!}

Of course black has a strong initiative if \variation{16. Nxd4 Qxg5}

Instead: \mainline{16. Bxd8 Nxe2+ 17. Nxe2 Rxb2 18. Bxc7 Bxc7}

\chessboard[showmover=false,labelleft=false,labelright=false,labeltop=false,labelbottom=false]

\subsubsection{12. Nbd2}
\newchessgame[setfen={1rbqk2r/2p1nppp/p1Pb4/8/Q1p5/5N2/PP3PPP/RNB2RK1 w k - 1 12}, mover=w, moveid=12w]
\chessboard[showmover=false,labelleft=false,labelright=false,labeltop=false,labelbottom=false]

\mainline{12. Nbd2 O-O}
Aimining to recapture on c4 with the knight.

\chessboard[showmover=false,labelleft=false,labelright=false,labeltop=false,labelbottom=false]

Here if 
\variation{13. a3} then black can take on c6 with \variation{13... Be6 14. Nxc4 Bd5} and \variation{16...Bxc6}

\noindent Back to the mainline: \mainline{13. Nxc4 Rb4 14. Qc2 Bf5}
White can blunder the exchange here with
\variation{15. Qe2?? Bd3!}

Instead of that, better is \mainline{15. Qc3 Nd5 16. Qd4}

\chessboard[showmover=false,labelleft=false,labelright=false,labeltop=false,labelbottom=false]

The easiest way for equality is for black to trade the c4 knight

\mainline{16...Nb6 17. b3 Nxc4 18. bxc4}

\chessboard[showmover=false,labelleft=false,labelright=false,labeltop=false,labelbottom=false]

Here \variation{18...Bd3!?} is possible, with the familiar idea of \variation{19. Qxd3??, Bxh2+}

But I prefer \mainline{18... Be6} There is only one critical line:
\mainline{19. Bb2 f6 20. Rac1 Ra4}

\chessboard[showmover=false,labelleft=false,labelright=false,labeltop=false,labelbottom=false]

Here if \variation{21. Rfd1 Qe7} is fine for black

The critical move is \mainline{21. Ng5!}

\chessboard[showmover=false,labelleft=false,labelright=false,labeltop=false,labelbottom=false]

\mainline{21...Bxh2+! 22. Kxh2 Qxd4 23. Bxd4 fxg5} with an equal ending.

\subsubsection{12. a3}
White forcibly prevents Rb4

\newchessgame[setfen={1rbqk2r/2p1nppp/p1Pb4/8/Q1p5/P4N2/1P3PPP/RNB2RK1 b k - 0 12}, mover=b, moveid=12b]
\chessboard[showmover=false,labelleft=false,labelright=false,labeltop=false,labelbottom=false]

\mainline{12...O-O 13. Nbd2 Be6 14. Nxc4 Bd5} This idea again, going for Bxc6
\mainline{15. Nxd6 Qxd6}

\chessboard[showmover=false,labelleft=false,labelright=false,labeltop=false,labelbottom=false]

White has several moves, but I think they can all be met with Qg6. For example:
\mainline{16. Bf4 Qg6 17. Bg3 Bxf3 18. gxf3 Rxb2} and black is slightly better with \variation{19... Nf5} in the air

\chessboard[showmover=false,labelleft=false,labelright=false,labeltop=false,labelbottom=false]

\subsection{Exchange French with c4}
Of all the exchange french lines, this is the most interesting one.

\newchessgame[setfen={r1bqkbnr/ppp2ppp/2n5/3p4/2PP4/5N2/PP3PPP/RNBQKB1R b KQkq - 0 5}, mover=b, moveid=5b]
\chessboard[showmover=false,labelleft=false,labelright=false,labeltop=false,labelbottom=false]

\mainline{5...Bb4+}
\newchessgame[setfen={r1bqk1nr/ppp2ppp/2n5/3p4/1bPP4/2N2N2/PP3PPP/R1BQKB1R b KQkq - 2 6}, mover=b, moveid=6b]

\subsubsection{6. Nc3}

At high depths (50), Nge7 is slightly better than Nf6. It is also more popular

\mainline{6...Nge7}

\chessboard[showmover=false,labelleft=false,labelright=false,labeltop=false,labelbottom=false]

White has a wide choice on move 6. Let's cover Be2/Bd3, a3, and c5

\variation{7. a3 Bxc3+ 8. bxc3 O-O 9. Be2 dxc4 10. Bxc4 Nd5 11. O-O Nxc3 12. Qb3?!}

\newchessgame[setfen={r1bq1rk1/ppp2ppp/2n5/8/2BP4/PQn2N2/5PPP/R1B2RK1 b - - 1 12}, mover=b, moveid=12b]
\chessboard[showmover=false,labelleft=false,labelright=false,labeltop=false,labelbottom=false]

\variation{12...Ne4 13. Re1? Na5! 14. Qc2 Nxc4 15. Qxc4 Nd6} and black is up a clean pawn

\newchessgame[setfen={r1bqk2r/ppp1nppp/2n5/3p4/1bPP4/2N2N2/PP3PPP/R1BQKB1R w KQkq - 3 7}, mover=w, moveid=7w]
\chessboard[showmover=false,labelleft=false,labelright=false,labeltop=false,labelbottom=false]

\mainline{7. Be2 dxc4}
The capture has to be played at some point anyways, and doing it now means Be2 and Bd3 will transpose immediately

\mainline{8. Bxc4 O-O} White can try \variation{9. h3} but black is fine with either \variation{9...Nd5} or \variation{9...Nf5}

\mainline{9. O-O Bg4} black threatens the d4 pawn. If \variation{10.d5?? Ne5} is good for black, he wins the d5 pawn.

Most popular is \mainline{10. Be3} when \mainline{10...Nf5 11. h3 Bh5!} keeping the tension scores 65\% for black on lichess.

I will show the full line because there is a tactic at the end for black to win, although without it he is also fine

\chessboard[showmover=false,labelleft=false,labelright=false,labeltop=false,labelbottom=false]

\mainline{12. g4 Nxe3 13. fxe3 Bg6 14. Nd5}

\chessboard[showmover=false,labelleft=false,labelright=false,labeltop=false,labelbottom=false]

\mainline{14...Na5!} and black is winning (attack on c4 and d5)



\subsubsection{7. Bd2}
\newchessgame[setfen={r1bqk1nr/ppp2ppp/2n5/3p4/1bPP4/5N2/PP1B1PPP/RN1QKB1R b KQkq - 2 6}, mover=b, moveid=6b]

\chessboard[showmover=false,labelleft=false,labelright=false,labeltop=false,labelbottom=false]

\mainline{6...Bd2}
\mainline{7. Qxd2 Bg4!}

If \variation{8. Be2} then \variation{8...dxc4} is good for black

The more difficult move is 

\mainline{8. cxd5 Qxd5 9. Nc3 Qe6+ 10. Be2 Bxf3 11. gxf3 O-O-O}

\chessboard[showmover=false,labelleft=false,labelright=false,labeltop=false,labelbottom=false]

\subsection{Popular 5. Bd3}
\newchessgame[setfen={r1bqkbnr/ppp2ppp/2n5/3p4/3P4/3B1N2/PPP2PPP/RNBQK2R b KQkq - 3 5}, mover=b, moveid=5b]
\chessboard[showmover=false,labelleft=false,labelright=false,labeltop=false,labelbottom=false]

White plays in the most natural and boring way. Black's plan is to play Nge7, Bf5, and Nce7.

\mainline{5...Bd6 6. O-O Nge7}

White has several options on move 7.

\mainline{7. c3 Bf5 8. Bxf5 Nxf5}

\variation{9. Qb3!? b6 10. Qb5 Qd7 11. Re1+ Nce7}=

\mainline{9. Re1+ Nce7 10. Bg5 f6 11. Bh4 g5 12. Bg3 Nxg3 13. hxg3 Qd7 14. Nbd2 O-O-O}
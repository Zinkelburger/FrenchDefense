\section{Exchange Variation}

The Exchange French has a few different move orders, but the main one is

\newchessgame
\noindent \mainline{1. e4 e6 2. d4 d5 3. exd5 exd5}

\chessboard[showmover=false,labelleft=false,labelright=false,labeltop=false,labelbottom=false]

\noindent White has a few different options here, but there are only two kinds of approaches: white plays for c4, or white plays without c4.

\subsection{White plays without c4}

\noindent Here white has two options, both equally unexciting

\subsection{4. Nf3}
\mainline{4. Nf3} The most common move in the position. The best scoring move and the one I recommend is \variation{4...Nc6}.

\chessboard[showmover=false,labelleft=false,labelright=false,labeltop=false,labelbottom=false]

Black wants to play Bd6, Nge7, Bf5, and Nxf5. However, if black starts with Bd6 then white 
can play \variation{5. c4}. Black is no longer able to play \variation{5...Bb4+} in one move. 
So \variation{5...Nc6} is like a waiting move.

White can play \variation{5. Bb5}, \variation{5. Bd3}, \variation{5. Bf4}, or \variation{5. Nc3}

\subsection{5. Bb5}
\variation{5. Bb5}:
\newchessgame[setfen={r1bqkbnr/ppp2ppp/2n5/1B1p4/3P4/5N2/PPP2PPP/RNBQK2R b KQkq - 3 5}, mover=b, moveid=5b]

\noindent White prepares to castle and pins the c6 knight. Black should do likewise and prioritize castling.

\mainline{5...Bd6}

\chessboard[showmover=false,labelleft=false,labelright=false,labeltop=false,labelbottom=false]

\noindent This is the best move, preparing Nge7.
White's only try is the interesing option of 
\mainline{6. c4!? dxc4 7. d5}

\noindent Of course black should play \mainline{7...a6} and if \variation{8. Bxc4} black can play Qe7+ when black either trades queens or plays Ne5.

\mainline{8. Ba4 b5}

\mainline{9. dxc6 bxa4}

\chessboard[showmover=false,labelleft=false,labelright=false,labeltop=false,labelbottom=false]

Here if \variation{10. Qxa4?!} black is better with  {Qe7+ 11. Be3 Bc5 12. O-O Bxe3}

\mainline{10. O-O Nge7}

\mainline{11. Nbd2}
hits c4

\mainline{11...Be6}
Defends c4

\chessboard[showmover=false,labelleft=false,labelright=false,labeltop=false,labelbottom=false]

\mainline{12. Qxa4 c3!}
Bd5 planning Bxc6 is the more popular line, but white is slightly better there. 
It is better to throw in c3 and only then play Bd5.

\mainline{13. bxc3 Bd5 14. Nd4 O-O}

Here \variation{15. c4?!} is the most popular move. 

But black is better with:

\chessboard[showmover=false,labelleft=false,labelright=false,labeltop=false,labelbottom=false]

\variation{15...Bxc6 16. Nxc6 Nxc6 17. Qxc6?? Qf6!}
And the dual threats of Bxh2+ and Qxa1 mean black is winning.

\variation{13. Ne4} is also met by \variation{13...Bd5}


\variation{11. Qxa4 Rb8}

\newchessgame[setfen={1rbqk2r/2p1nppp/p1Pb4/8/Q1p5/5N2/PP3PPP/RNB2RK1 w k - 1 12}, mover=w, moveid=12w]
\chessboard[showmover=false,labelleft=false,labelright=false,labeltop=false,labelbottom=false]

Qxc4, Nbd2, and a3
\subsubsection{12. Qxc4?!}
\mainline{12. Qxc4?! O-O 13. Nc3?! Rb4 14. Qe2 Nxc6 15. Bg5}

\chessboard[showmover=false,labelleft=false,labelright=false,labeltop=false,labelbottom=false]

Here black has only played \variation{15...f6?} on lichess
But black can get a sizeable advantage (-1.1) with 
\mainline{15...Nd4!}

If \variation{16. Bxd8 Nxe2+ 17. Nxe2 Rxb2 18. Bxc7 Bxc7}
Of course black has a strong initiative if \mainline{16. Nxd4 Qxg5}

\chessboard[showmover=false,labelleft=false,labelright=false,labeltop=false,labelbottom=false]

\subsubsection{12. Nbd2}
\newchessgame[setfen={1rbqk2r/2p1nppp/p1Pb4/8/Q1p5/5N2/PP3PPP/RNB2RK1 w k - 1 12}, mover=w, moveid=12w]


\subsection{c4}